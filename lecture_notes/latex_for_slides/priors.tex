\documentclass{beamer}
\usepackage{amsmath}
\usepackage{tikz}
\usetikzlibrary{bayesnet}

\title{Bernstein--von Mises Theorem}
\author{Prof. Dr. Oliver Dürr}
\institute{HTWG Konstanz -- Bayesian Data Analysis (VI)}
\date{}

\begin{document}

\begin{frame}
  \titlepage
\end{frame}

%------------------------------------------------
\begin{frame}{Bernstein --von Mises Theorem}

Let \( y_1, \dots, y_n \sim p(y \mid \theta) \) i.i.d. from a statistical model with true parameter \( \theta_0 \), and suppose:
\begin{itemize}
  \item The prior \( p(\theta) \) is continuous and \alert{positive} near \( \theta_0 \), and the model satisfies mild regularity conditions.
\end{itemize}

\vspace{.25em}

Then, as \( n \to \infty \), the posterior
\[
p(\theta \mid y_{1:n}) \propto p(\theta) \cdot \underbrace{\prod_{i=1}^n p(y_i \mid \theta)}_{\text{likelihood}}
\]
converges \textcolor{blue}{(CLT-like, at rate \(1/\sqrt{n}\))} to
\[
\mathcal{N}\left(\hat{\theta}_{\text{MLE}}, \frac{1}{n \cdot I(\theta_0)} \right),
\]
where \( \hat{\theta}_{\text{MLE}} \) is the value that maximizes the likelihood, and
\[
I(\theta_0) = - \mathbb{E}\left[ \frac{\partial^2}{\partial \theta^2} \log p(y \mid \theta) \right] \Bigg|_{\theta = \theta_0}
\;\; \text{(Fisher Information)}
\]


\end{frame}


\begin{frame}{Intuition and Implications}
\vspace{-0.5em}
\begin{itemize}
  \item As \( n \to \infty \), the data dominate the prior.
  \item The posterior becomes approximately normal.
  \item CLT like behavior $\propto 1/\sqrt{n}$. 
  \item Be careful to allow for the prior to be positive near \( \theta_0 \)
  \item Bayesian inference $\approx$ Frequentist inference in large samples.
\end{itemize}
\vspace{1em}
For small \( n \), you can only do better (or worse) than the frequentist approach. 
\end{frame}

%------------------------------------------------

\end{document}